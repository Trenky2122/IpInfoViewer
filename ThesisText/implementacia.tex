\chapter{Implementácia}

\label{kap:implementacia}

Systém je implementovaný ako súbor programov bežiacich v kontaineroch. Každý program má svoju úlohu.
\begin{enumerate}
    \item Open Street Maps tile server
    \item Webový server pre frontend
    \item Webový server pre backed
    \item Konzolová aplikácia Seed
    \item Background service IpInfoMap
    \item Background service CountryPingInfo
\end{enumerate}

\section{Nástroje používané na implementáciu}

Na písanie kódu sa dajú používať rôzne nástroje v závislosti od jazyka, v ktorom chceme kód písať. Na písanie backendového kódu v C\# 
sa najčastejšie používajú nástroje Visual Studio alebo Rider. Vzhľadom na osobnú preferenciu sme sa rozhodli použiť nástroj Visual Studio 2022, 
ktorého Community verzia je zadarmo na stiahnutie.

Na písanie frontendového kódu v TypeScript, HTML a CSS sa najčastejšie používajú nástroje Visual Studio Code, WebStorm alebo aj rôzne iné. 
Vzhľadom na osobnú preferenciu sme sa rozhodoli použiť WebStorm, na ktorý máme udelenú študentskú licenciu.

\section{Štruktúra backendového riešenia}

V jazyku C\# je možnosť zoskupovať viacero aplikácií do jedného riešenia. V praxi to vyzerá tak, že riešenie je znázornené ako súbor .sln a každá 
aplikácia alebo knižnica je znázornená ako súbor .csproj. Aplikáciám a knižniciam sa vnútorne hovorí projekty. V našom riešení sa nachádzajú nasledovné 
projekty:

\begin{itemize}
    \item IpInfoViewer.Web
    - projekt, v ktorom je obsiahnuté riešenie frontendovej časti aplikácie
    \item IpInfoViewer.Api
    - projekt slúžiaci ako API pre komunikáciu medzi backendom a frontendom
    \item IpInfoViewer.IpInfoService
    - projekt, ktorý slúži ako servis bežiaci na pozadí ktorý vytvorí v lokálnom dátovom sklade zoznam všetkých ip adries s ich lokalitami a krajinami
    \item IpInfoViewer.CountryPingInfoService
    - projekt, ktorý slúži ako servis bežiaci na pozadí spracovávajúci údaje v týždenných intervaloch o priemernom pingu pre danú krajinu
    \item IpInfoViewer.MapPointsService
    - projekt, ktorý slúži ako servis bežiaci na pozadí spracovávajúci údaje v týždenných intervaloch o priemernom pingu pre dané geografické 
    okolie (zemepisnú šírku a dĺžku)
    \item IpInfoViewer.Libs
    - knižnica obsahujúca väčšinu aplikačnej logiky, tak aby bola zahrnuteľná zo všetkých ostatných projektov
\end{itemize}

\subsection{IpInfoViewer.Api}
Tento projekt je ASP.NET Core 6.0 aplikácia bežiaca na webovom serveri Kestrel. Obsahuje nasledujúce triedy: 
\begin{itemize}
    \item Program
    - Základná trieda každého spustiteľného programu. Jej metóda Main je vstupným bodom každého C\# programu, teda po spustení programu sa automaticky 
    začne vykonávať. Registrujú sa v nej triedy do DI kontainera (Dependency Injection), z ktorého sa potom vyberajú z iných tried vrámci behu programu. 
    Tiež tu prebieha inicializácia potrebných nastavení programu. Od verzie .NET 6.0 je podporovaná upravená syntax tejto triedy na tzv. top-level statements. 
    To znamená, že v súbore Program.cs je vynechaná inicializácia triedy Program a metódy Main, nakoľko bola v každom 
    programe rovnaká. Vnútorne sú však tieto triedy stále používané, ide len o syntaktickú skratku. Používanie top-level statements 
    je dobrovoľné, vrámci čistoty kódu ich ale využívame.
    \begin{lstlisting}[float,language={[Sharp]C},caption={Ukážka kódu triedy Program bez použitia top-level statements},label=alg:Program_Old]
        class Program 
        {
            public static void Main(string[] args) 
            {
                System.Console.WriteLine("Hello World");
            }
        }
    \end{lstlisting}
    \begin{lstlisting}[float,language={[Sharp]C},caption={Ukážka kódu triedy Program s použitím top-level statements},label=alg:Program_New]
        System.Console.WriteLine("Hello World");
    \end{lstlisting}
    \item MapPointsController
    - Trieda slúži ako súbor metód, ktoré sa dajú volať cez protokol HTTP z frontendovej časti aplikácie. Obsahuje metódy pre získavanie 
    dát zoskupených podľa zemepisnej dĺžky a šírky.
    \item CountryPingInfoController
    - Súbor metód volateľných cez HTTP. Obsahuje metódy pre dáta zoskupené podľa krajiny.
\end{itemize}
\subsection{IpInfoViewer.IpInfoService}
Projekt je konzolová aplikácia bežiaca ako servis na pozadí. Obsahuje nasledovné triedy:
\begin{itemize}
    \item Program
    \item IpInfoServiceWorker
    - Treida reprezentujúca samotný servis. Trieda je zadefinovaná v triede Program ako "Hosted Service", teda pri spustení sa vykoná jej metóda ExecuteAsync. 
    Načíta súbor s priradením medzi IP adresou a geografickými dátami a paralelne pošle každý jeho riadok na spracovanie.
\end{itemize}
\subsection{IpInfoViewer.CountryPingInfoService}
Projekt je rovnako konzolová aplikácia bežiaca ako servis na pozadí. Obsahuje nasledovné triedy:
\begin{itemize}
    \item Program
    \item CountryPingInfoServiceWorker
    - Hosted service tohto servisu.
    Načita všetky známe IP adresy spracované servisom IpInfoService. Následne ich zoskupí podľa krajín prislúchajúcich daným adresám. 
    Nakoniec ešte zistí, kedy boli dáta naposledy spracovávané a pošle na spracovanie dáta paralelne pre každý týždeň od posledného spracovania 
    až po súčasnosť.
\end{itemize}
\subsection{IpInfoViewer.MapPointsService}
Taktiež servis na pozadí. Obsahuje nasledovné triedy:
\begin{itemize}
    \item Program
    \item MapPointsServiceWorker
    - Hosted service tohto servisu.
    Funguje rovnako ako CountryPingInfoServiceWorker, ale adresy zoskupuje podľa geografických súradníc. Adresy sú rozdelené do stabilných sektorov 
    na mape. Mapa je rozdelená do 23x60 sektorov rovnomerne podľa geografickej šírky a dĺžky. 
\end{itemize}
\subsection{IpInfoViewer.Libs}
Knižnica zoskupujúca logiku všetkých C\# programov. Ostatné programy ju využívajú ako referenciu. Je tomu tak preto, lebo programy navzájom zdieľajú 
časť logiky a bolo by nesprávne túto logiku písať do každého programu zvlášť. Tiež sme sa chceli vyhnúť referencovaniu ostatných spustiteľných programov navzájom 
priamo, pretože sa to ukázalo ako nepraktické. Dôvodov je niekoľko, najvážnejší s ktorým sme sa stretli je konflikt mien konfiguračných súborov. Pri C\# 
projektoch je zaužívané používať konfiguračné súbory s menami appsettings.json a appsettings.{environmentName}.json. Tiež je pravidlo, že keď referencujeme 
iný projekt, do priečinka s výstupom kompilácie idú všetky súbory rovnako, ako keď kompilujeme projekt samotný. Keďže aj pôvodný projekt aj referencovaný projekt 
obshujú súbory appsettings.json, nastane konflikt a do výstupného priečinka sa nakopíruje len jeden z nich, a podľa skúseností to bol vždy ten z referencovaného projektu, 
čo je ten nesprávny. Tomuto sa dá vyhnúť tým, že spustiteľné projekty sa navzájom nereferencujú a namiesto toho všetky referencujú jednu knižnicu. Knižnica nemá takéto 
konfiguračné súbory, preto ku konfliktu nedochádza. Knižnica IpInfoViewer.Libs obsahuje nasledovné triedy: 
\begin{itemize}
    \item BaseModel
    - Abstraktná trieda z ktorej dedia všetky modely, teda triedy predstavujúce dáta. Obsahuje všetky parametre ktoré by mali zdieľať všetky modely v riešení. V súčasnosti 
    je jediným takýmto parametrom Id, teda primárny kľúč v databázovej tabuľke, v našom riešení znázornené ako 64-bitové celé číslo.
    \item BaseWeeklyProcessedModel
    - Abstraktná trieda, ktorá je rozšírením triedy BaseModel o vlastnosti ValidFrom a ValidTo reprezentované časovou pečiatkou. Tieto vlastnosti 
    sa opakovali vo všetkých časovo spracovávaných triedach.
    \item BaseMapModel 
    - Abstraktná trieda ktorá je rozšírením BaseMapModel o vlastnosti IpAddressesCount a AveragePingRtT. Ide o pre nás zaujímavé vlastnosti ktoré sa vyskytovali vo 
    viacerých modeloch. Prvá znázorňuje počet IP adries ktoré daný záznam znázorňuje, druhá ich priemerný ping v milisekundách.
\end{itemize}