\chapter{Možnosti implementácie}

\label{kap:moznosti_implementacie}

Každý softvér si musí určiť, pre aké platformu je vyvíjaný. Samozrejme, čím väčšiu škálu platforiem pokryje,
tým je dostupnejší pre väčšie množstvo používateľov. Dnes drvivá väčšina používateľov používa istú podmnožinu
zo systémov Android, Linux, Windows, iOS a MacOS X. Spoločnou črtou týchto platforiem je, že medzi nimi neexistuje taká dvojica,
z ktorej by jedna platforma natívne podporovala programy napísané pre druhú platformu. Vyvíjať aplikáciu päťkrát,
pre každú platformu v jej natívnom prostredí, však nie je veľmi efektívne. 

Namiesto toho dnes existuje niekoľko riešení umožňujúcich zdieľať časti zdrojového kódu medzi riešeniami pre rôzne systémy.
Ako príklad môžme uviesť frameworky Flutter, React Native alebo MAUI pre .NET.

Ako univerzálne riešenie sa dnes používajú webové aplikácie.
Takéto aplikácie majú niekoľko dôležitých výhod, ako napríklad rýchla dostupnosť, žiadna nutnosť inštalácie a jednoduché spustenie
na všetkých platformách, na ktorých je dostupný webový prehliadač (napríklad Firefox, Chrome alebo Opera).
Webové aplikácie zvyknú mávať dve oddelené časti - frontend, bežiaci vo webovom prehliadači klienta a backend,
bežiaci na serveri. Tieto dve časti spolu komunikujú pomocou siete, najčastejšie pomocou protokolov HTTP(S) alebo WebSocket.
Úlohou frontendu je poskytovať prívetivé rozhranie s ktorým môže používateľ pracovať, úlohou backendu je zase spracovávať dáta,
komunikovať s databázou a poskytovať dáta frontendu. Práve pre takýto model som sa rozhodol.

\section{Možnosti vývoja frontendového riešenia}
Programovanie webových stránok sa dnes nezaobíde bez programovacieho jazyka JavaScript, ktorý používajú webové prehliadače. Alternatívne 
sa dajú použiť prekladače iných jazykov do WebAssembly (napríklad framework Blazor pre .NET) alebo jazyk TypeScript, ktorý je rozšírením JavaScriptu o typový 
systém a iné užitočné konštrukcie, zároveň je plne kompatibilný s JavaScriptom (je jeho nadmnožinou) a vo webových prehliadačoch sa spúšťa pomocou 
prekladu do JavaScriptu. Na vývoj webového frontendu je možnosť používať webové frameworky, ktoré umožňujú delenie stránok na moduly, 
ktoré sa dajú jednotlivo znovupoužívať a paranetrizovať na rôznych podstránkach aplikácie, čím zjednodušujú spravovanie a čitateľnosť aplikácie. 
V súčasnej dobe sú najpopulárnejšie frameworky React, Vue a Angular. Kým React a Vue umožňujú prácu v JavaScripte ale volitelne je možné vyvíjať 
aj v TypeScripte, Angular prácu v TypeScripte vyžaduje. Tieto frameworky sú podporované vo všetkých nových verziách moderných prehliadačov. 
Keďže všetky tieto frameworky fungujú na veľmi podobnom princípe, je veľmi ťažké povedať, ktorý by bola pre moju aplikáciu najvhodnejší. Vzhľadom 
na to som sa rozhodol pre Angular, v ktorom mám naviac praktických skúseností.

\section{Možnosti vývoja backendového riešenia}
Pre vývoj backendového riešenia je vhodné vybrať taký programovací jazyk, pre ktorý existuje framework umožňujúci vývoj API, ktoré bude podporovať 
protokol HTTP(s). Takýchto jazykov existuje veľmi veľa, dokonca pre niektoré jazyky existuje viacero rôznych knižníc umožňujúcich výstavbu 
takýchto API. Ako príklad môžme uviesť jazyk Python a knižnice FastAPI alebo Flask a mnohé iné, jazyk Java a framework Spring Boot, jazyk C\# a 
framework ASP.NET Core a mnohé iné. Všetky tieto možnosti majú dostatočné nástroje a je ťažké určiť, ktorý z nich je objektívne najvhodnejší, 
preto som sa rozhodol pre jazyk C\# a ASP.NET Core, s ktorým mám najviac skúseností.

\subsection{Možnosti implementácie vo vybranom jazyku}