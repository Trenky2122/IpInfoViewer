\chapter{Možnosti implementácie}

\label{kap:moznosti_implementacie}

Každý softvér si musí určiť, pre aké platformu je vyvíjaný. Samozrejme, čím väčšiu škálu platforiem pokryje,
tým je dostupnejší pre väčšie množstvo používateľov. Dnes drvivá väčšina používateľov používa istú podmnožinu
zo systémov Android, Linux, Windows, iOS a MacOS X. Spoločnou črtou týchto platforiem je, že medzi nimi neexistuje taká dvojica,
z ktorej by jedna platforma natívne podporovala programy napísané pre druhú platformu. Vyvíjať aplikáciu päťkrát,
pre každú platformu v jej natívnom prostredí, však nie je veľmi efektívne. 

Namiesto toho dnes existuje niekoľko riešení umožňujúcich zdieľať časti zdrojového kódu medzi riešeniami pre rôzne systémy.
Ako príklad môžme uviesť frameworky Flutter, React Native alebo MAUI pre .NET.

Ako univerzálne riešenie sa dnes používajú webové aplikácie.
Takéto aplikácie majú niekoľko dôležitých výhod, ako napríklad rýchla dostupnosť, žiadna nutnosť inštalácie a jednoduché spustenie
na všetkých platformách, na ktorých je dostupný webový prehliadač (napríklad Firefox, Chrome alebo Opera).
Webové aplikácie zvyknú mávať dve oddelené časti - frontend, bežiaci vo webovom prehliadači klienta a backend,
bežiaci na serveri. Tieto dve časti spolu komunikujú pomocou siete, najčastejšie pomocou protokolov HTTP(S) alebo WebSocket.
Úlohou frontendu je poskytovať prívetivé rozhranie s ktorým môže používateľ pracovať, úlohou backendu je zase spracovávať dáta,
komunikovať s databázou a poskytovať dáta frontendu. 